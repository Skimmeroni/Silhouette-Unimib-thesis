\documentclass{beamer}
\usepackage{amsmath}
\usepackage{bookmark}
\usepackage{booktabs}
\usepackage{pgfplotstable}

\usetheme{Rochester}

\title{Studio e approfondimento del coefficiente Silhouette per la
valutazione interna di risultati di clustering non-supervisionato}
\author{XXX YYY ZZZZZZ}
\date{\today}

\begin{document}

    \begin{frame}
        \titlepage
    \end{frame}

    %\begin{frame}
    %    \frametitle{Outline}
    %    \tableofcontents
    %\end{frame}

    \section{Clustering}

        \begin{frame}
            \frametitle{Clustering}

            Da dizionario Merriam-Webster:

            \vfill

            \begin{block}{Cluster}
                \emph{``A number of similar things that occur together''}
            \end{block}

            \vfill

            \begin{block}{Cluster analysis}
                \emph{``A statistical classification technique for discovering
                      whether the individuals of a population fall into different groups
                      by making quantitative comparisons of multiple characteristics''}
            \end{block}
        \end{frame}

        \begin{frame}
            \frametitle{Quanti sono i cluster?}

            \begin{itemize}
                \item
                Gli algoritmi di clustering non sono in grado di fornire una
                metrica oggettiva per determinare quale sia il corretto numero
                di cluster
                \item
                Questo é particolarmente problematico negli algoritmi che hanno
                il numero di cluster come iperparametro (e.g. K-Means)
                \item
                Nel clustering non supervisionato non vi è a disposizione
                alcuna "ground truth"
                \item
                Un numero di cluster errato induce un raggruppamento non
                rappresentativo del dataset, bensí un raggruppamento artificioso
            \end{itemize}
        \end{frame}

        \begin{frame}
            \frametitle{Introduzione di Silhouette}
            Silhouette si propone di rispondere alle seguenti domande:

            \begin{itemize}
                \item
                Il clustering è di buona qualità?
                \item
                Quali elementi sono stati ben classificati?
                \item
                Quali elementi sono inclassificabili?
                \item
                Il numero di cluster scelto è rappresentativo del dataset?
            \end{itemize}
        \end{frame}

    \section{Silhouette}

        \begin{frame}
            \frametitle{Situazione iniziale}

            \begin{itemize}
                \item
                Si consideri un dataset di dimensione $N \times M$
                ($N$ elementi, $M$ attributi)
                \item
                Si assuma che gli attributi siano tutti dati numerici
                (altezze, lunghezze, ecc...) e che siano tutti noti
                \item
                Da questo é possibile costruire una \textbf{matrice
                delle distanze} $N \times N$
                \item
                La cella $(i, j)$ contiene il valore $d(i, j)$, che
                rappresenta il grado di "dissomiglianza" fra $i$ e $j$
            \end{itemize}
        \end{frame}

        \begin{frame}
            \frametitle{Funzione di distanza}

            Il grado di dissomiglianza fra $i$ e $j$ è calcolato mediante
            una \textbf{funzione di distanza} applicata ai loro attributi.
            Ad esempio:

            \vfill

            \begin{block}{Distanza Euclidea}
                \begin{equation*}
                    d(i, j) = \sqrt{\sum_{m = 1}^{M} (f_{i, m} - f_{j, m})^{2}}
                \end{equation*}

            Con $f_{i, m}$, $f_{j, m}$ valori del $m$-esimo attributo
            \end{block}

            \vfill

            Altri esempi: \textbf{distanza di Manhattan},
            \textbf{distanza di Minkowski}.
        \end{frame}

        \begin{frame}
            \frametitle{K-Means}

            \begin{table}[h]
            \centering
                \pgfplotstabletypeset[
                col sep=comma,
                header=true,
                every head row/.style={before row=\toprule, after row=\midrule},
                every last row/.style={after row=\bottomrule},
                ]{data/sdist.csv}
                \caption{Matrice delle distanze per il dataset \texttt{iris} (solo i primi 6 elementi).}
                \label{tab:dist}
            \end{table}

            \vfill

            Nota la matrice delle distanze, si applichi ad esempio K-Means
            per suddividere il dataset in $K$ cluster
        \end{frame}

        %\begin{frame}
            %\frametitle{}
        %\end{frame}

    %\section{Scelta dei pacchetti}

    %\section{Uso di Silhouette su EHR}

    %\section{Aspetti teorici di Silhouette}

    %\section{Dataset reali}

    \section{Fine}

        \begin{frame}
            \frametitle{Fine}
            \centering
            \Huge{Domande?}

            \includegraphics{Chill.png}
        \end{frame}

\end{document}
